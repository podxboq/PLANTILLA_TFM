\section{Contexto y estado del arte}

Después de la introducción, se suele describir el contexto de aplicación. Suele ser un capítulo (o dos en ciertos casos) en el que se estudia a fondo el dominio de aplicación, citando numerosas referencias. Debe aportar un buen resumen del conocimiento que ya existe en el campo de los problemas habituales identificados.

Es conveniente que revises los estudios actuales publicados en la línea elegida, y deberás consultar diferentes fuentes. No es suficiente con la consulta on-line, es necesario acudir a la biblioteca y consultar manuales técnicos.

Hay que tener presente los autores de referencia en la temática del trabajo de investigación. Si se ha excluido a alguno de los relevantes hay que justificar adecuadamente su exclusión. Si por la extensión del trabajo no se puede señalar a todos los autores, habrá que justificar por qué se han elegido unos y se ha prescindido de otros.

La organización específica en secciones dependerá estrechamente el trabajo concreto que vayas a realizar. En este punto será fundamental la colaboración con tu DIRECTOR, él podrá asesorarte y guiarte, aunque siempre debes tener claro que el trabajo fundamental es tuyo.

El capítulo debería concluir con una última sección de resumen de conclusiones, resumiendo las principales averiguaciones del estudio del dominio y cómo van a afectar al desarrollo específico del trabajo.

Recuerda que para citar trabajos de diferentes autores es fundamental e imprescindible seguir el formato APA, según se describe en el documento Normativa_APA.pdf disponible en el apartado de Documentación del Aula de información general del Máster Universitario en Computación Cuántica (MUCC). No se debe mencionar, ni utilizar ninguna fuente, sin citarla apropiadamente.

\subsection{Objetivos y metodología de trabajo}

Este tercer capítulo es el puente entre el estudio del dominio y la contribución a realizar. Según el trabajo concreto, el bloque se puede organizar de distintas formas, pero hay tres elementos que deben estar presentes con mayor o menor detalle: (1) objetivo general, (2) objetivos específicos y (3) metodología de trabajo.

Es muy importante, por no decir imprescindible, que los objetivos (general y específicos) sean SMART (Doran, 1981) según la idea de George T. Doran que utilizó la palabra smart (inteligente en inglés) para definir las características de un objetivo:
\begin{itemize}
	\item[S] Specific / Específico: que exprese claramente qué es exactamente lo que se quiere conseguir.
	\item[M] Measurable / Medible: que se puedan establecer medidas que determinen el éxito o fracaso y también el progreso en la consecución del objetivo.
	\item[A] Attainable / Alcanzable: que sea viable su consecución en base al esfuerzo, tiempo y recursos disponibles para conseguirlo.
	\item[R] Relevant / Relevante: que tenga un impacto demostrable, es decir que sea útil para un propósito concreto.
	\item[T] Time-Related / Con un tiempo determinado: que se pueda llevar a cabo en una fecha determinada.
\end{itemize}

\subsubsection{Objetivo general}

Los trabajos aplicados se centran en conseguir un impacto concreto, demostrando la efectividad de una tecnología, proponiendo una nueva metodología o aportando nuevas herramientas tecnológicas. El objetivo por tanto no debe ser sin más “crear una herramienta” o “proponer una metodología”, sino que debe centrarse en conseguir un efecto observable. Además, como se ha dicho antes el objetivo general debe ser SMART.

Ejemplo de objetivo general SMART: Mejorar el servicio de audio guía de un museo convirtiéndolo en una guía interactiva controlada por voz y valorada positivamente, un mínimo 4 sobre 5, por los visitantes del museo.

Este objetivo descrito anteriormente podría dar lugar a un trabajo de tipo 2 (desarrollo de software) que plantease el desarrollo de un bot conversacional que procesara la señal de voz recogida a través del micrófono y a través de técnicas de procesamiento del lenguaje natural fuera capaz de mantener una conversación con el visitante para determinar el contenido en el que está interesado o resolver las posibles dudas o preguntas que pudiera tener a lo largo de su visita.

\subsubsection{Objetivos específicos}

Independientemente del tipo de trabajo, la hipótesis o el objetivo general típicamente se dividirán en un conjunto de objetivos específicos analizables por separado. Estos objetivos específicos suelen ser explicaciones de los diferentes pasos o tareas a seguir en la consecución del objetivo general.

Con los objetivos específicos has de concretar qué pretendes conseguir. Estos objetivos que deben ser SMART se formulan con un verbo en infinitivo más el contenido del objeto de estudio. Se suelen usar viñetas para cada uno de los objetivos. Se pueden utilizar fórmulas verbales, como las siguientes:
\begin{itemize}
	\item Analizar
	\item Calcular
	\item Clasificar
	\item Comparar
	\item Conocer
	\item Cuantificar
	\item Desarrollar
	\item Describir
	\item Descubrir
	\item Determinar
	\item Establecer
	\item Explorar
	\item Identificar
	\item Indagar
	\item Medir
	\item Sintetizar
	\item Verificar
\end{itemize}

Los objetivos específicos suelen ser alrededor de 5. Normalmente uno o dos sobre el marco teórico o estado del arte y dos o tres sobre el desarrollo específico de la contribución.

En un trabajo como el anterior se incluirían objetivos específicos tales como:
\begin{itemize}
\item Identificar las tecnologías disponibles para crear un chatbot
\item Explorar recursos lingüísticos online que describan el dominio de los muesos en español
\item Diseñar e implementar el módulo de gestión del dialogo
\item Evaluar el agente conversacional con 10 visitantes del museo.
\end{itemize}

\subsection{Metodología del trabajo}

De cara a alcanzar los objetivos específicos (y con ellos el objetivo general o la validación/refutación de la hipótesis), será necesario realizar una serie de pasos. La metodología del trabajo debe describir qué pasos se van a dar, el porqué de cada paso, qué instrumentos se van a utilizar, cómo se van a analizar los resultados, etc.

\newpage