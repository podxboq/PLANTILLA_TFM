\section{Introducción}

El primer capítulo es siempre una introducción. En ella debes resumir de forma esquemática pero suficientemente clara lo esencial de cada una de las partes del trabajo. La lectura de este primer capítulo ha de dar una primera idea clara de lo que se pretendía, las conclusiones a las que se ha llegado y del procedimiento seguido.

Como tal, es uno de los capítulos más importantes de la memoria. Las ideas principales a transmitir son la identificación del problema a tratar, la justificación de su importancia, los objetivos generales (a grandes rasgos) y un adelanto de la contribución que esperas hacer.

Típicamente una introducción tiene tres apartados: Motivación, Planteamiento del trabajo, Estructura del trabajo. (Texto Normal del menú de estilos.)

(Ejemplo de nota al pie\footnote{Ejemplo de nota al pie.}.)

\subsection{Motivación}

¿Cuál es el problema que quieres tratar?

¿Cuáles crees que son las causas?

¿Por qué es relevante el problema? (Texto Normal del menú de estilos.)

A continuación, se indica con un ejemplo cómo deben introducirse los títulos y las fuentes en Tablas y Figuras.