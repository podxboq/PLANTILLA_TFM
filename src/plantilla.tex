\documentclass[11pt,a4paper,spanish]{book}
\usepackage{estilo_unir-1}



%---------------------------
%título del trabajo y autor
%---------------------------
\title{Escribir el título del documento}
\titulacion{Máster en Computación Cuántica}
\author{Nombre y Apellidos}
\date{d de mes de 2022}
\director{ nombre y apellidos}
\nombreciudad{Nombre ciudad}

%---------------------------
%marges
%---------------------------
%\usepackage[margin=1.9cm]{geometry}
%---------------------------
%---------------------------
%---------------------------
%---------------------------
\begin{document}
\renewcommand{\listfigurename}{Índice de Ilustraciones}
\renewcommand{\listtablename}{Índice de Tablas}
\renewcommand{\contentsname}{Índice de Contenidos}
\renewcommand{\figurename}{Figura}
\renewcommand{\tablename}{Tabla} 

\maketitle

\frontmatter
\tableofcontents
\listoffigures
\listoftables

\chapter{Resumen}
{\bf Nota:} En este apartado se introducirá un breve resumen en español del trabajo realizado (extensión máxima: 150 palabras). Este resumen debe incluir el objetivo o propósito de la investigación, la metodología, los resultados y las conclusiones.


{\bf Palabras Clave:} Se deben incluir de 3 a 5 palabras claves en español

\chapter{Abstract}
{\bf Nota:} En este apartado se introducirá un breve resumen en español del trabajo realizado (extensión máxima: 150 palabras). Este resumen debe incluir el objetivo o propósito de la investigación, la metodología, los resultados y las conclusiones.


{\bf Palabras Clave:} Se deben incluir de 3 a 5 palabras claves en inglés




\mainmatter
\chapter{Introducción}

El primer capítulo es siempre una introducción. En la introducción se debe resumir de forma esquemática pero suficientemente clara lo esencial de cada una de las partes del trabajo. La lectura de este primer capítulo ha de dar una idea clara de lo que se pretendía, las conclusiones a las que se ha llegado y del procedimiento seguido.

Como tal, es uno de los capítulos más importantes de la memoria. Las ideas principales a transmitir son la identificación del problema a tratar, la justificación de su importancia, los objetivos generales a grandes rasgos y un adelanto de la contribución que esperas hacer.

Típicamente una introducción tiene tres apartados:
\begin{itemize}
\item Motivación / justificación del tema a tratar
\item Planteamiento del trabajo
\item Estructura del trabajo
\end{itemize}


El primer capítulo es siempre una introducción. En la introducción se debe resumir de forma esquemática pero suficientemente clara lo esencial de cada una de las partes del trabajo. La lectura de este primer capítulo ha de dar una idea clara de lo que se pretendía, las conclusiones a las que se ha llegado y del procedimiento seguido.

Como tal, es uno de los capítulos más importantes de la memoria. Las ideas principales a transmitir son la identificación del problema a tratar, la justificación de su importancia, los objetivos generales a grandes rasgos y un adelanto de la contribución que esperas hacer.



El primer capítulo es siempre una introducción. En la introducción se debe resumir de forma esquemática pero suficientemente clara lo esencial de cada una de las partes del trabajo. La lectura de este primer capítulo ha de dar una idea clara de lo que se pretendía, las conclusiones a las que se ha llegado y del procedimiento seguido.

Como tal, es uno de los capítulos más importantes de la memoria. Las ideas principales a transmitir son la identificación del problema a tratar, la justificación de su importancia, los objetivos generales a grandes rasgos y un adelanto de la contribución que esperas hacer.



El primer capítulo es siempre una introducción. En la introducción se debe resumir de forma esquemática pero suficientemente clara lo esencial de cada una de las partes del trabajo. La lectura de este primer capítulo ha de dar una idea clara de lo que se pretendía, las conclusiones a las que se ha llegado y del procedimiento seguido.

Como tal, es uno de los capítulos más importantes de la memoria. Las ideas principales a transmitir son la identificación del problema a tratar, la justificación de su importancia, los objetivos generales a grandes rasgos y un adelanto de la contribución que esperas hacer.



El primer capítulo es siempre una introducción. En la introducción se debe resumir de forma esquemática pero suficientemente clara lo esencial de cada una de las partes del trabajo. La lectura de este primer capítulo ha de dar una idea clara de lo que se pretendía, las conclusiones a las que se ha llegado y del procedimiento seguido.

Como tal, es uno de los capítulos más importantes de la memoria. Las ideas principales a transmitir son la identificación del problema a tratar, la justificación de su importancia, los objetivos generales a grandes rasgos y un adelanto de la contribución que esperas hacer.



\chapter{Contexto y Estado del Arte}

\chapter{Identificación de Requisitos}

\chapter{Objetivos}

\chapter{Desarrollo del trabajo}

\chapter{Conclusiones y Trabajo Futuro}

\begin{thebibliography}{a}
\bibitem{etiqueta} \textsc{Autores},
\textit{nombre referencia.}
Información addicional
\end{thebibliography}
%\bibliographystyle{plain} 
%\bibliography{bibliografia}

\appendix
\chapter{Apendices}

\end{document}





















