\chapter{Contexto y estado del técnica}

Después de la introducción, se suele describir el contexto de aplicación. En este Capítulo debemos mostrar un sobrado conocimiento de la materia, plagando absolutamente TODO lo que mencionemos con referencias. Práticamente, cada frase puede necesitar de alguna referencia.

Las referencias NO están solo para aparentar. Rendimos tributo y reconocemos a las personas que han pensado los problemas antes que nosotros. Puede haber varias referencias válidas para una misma afirmación, y no pasa nada porque así se indique.

Recuerda que para citar trabajos de diferentes autores es fundamental e imprescindible seguir el formato APA, según se describe en el documento Normativa\_APA.pdf disponible en el apartado de Documentación del Aula de información general del Máster Universitario en Computación Cuántica (MUCC). No se debe mencionar, ni utilizar ninguna fuente, sin citarla apropiadamente.

EJEMPLO DE CITAS: Si queremos citar a alguien, por ejemplo porque vamos a hablar de Latex~\citep{lamport1994} o porque, según las ideas de~\cite{ackerman2017}, la liga de fútbol inglesa debe tener torneos de desempate, pues tenemos que hacerlo correctamente. \textbf{(Ver en la sección Bibliografía cómo deben incluirse las entradas bibliográficas)}.

Este Capítulo puede tener secciones diferentes a las que se indican a continuación, que se indican a modo de ejemplo.

\section{Antecedentes históricos}

\section{Estado actual}